\hypertarget{index_intro_sec}{}\section{Introduction to Voting System}\label{index_intro_sec}
Welcome to The \+Voting System Sotfware of C\+S\+CI 5801 -\/ Fall 2019 University of Minnesota The goal of this software project is be able to perform two types of Voting System\+: Open \hyperlink{classParty}{Party} List (\hyperlink{classOPL}{O\+PL}) and Closed \hyperlink{classParty}{Party} List (\hyperlink{classCPL}{C\+PL}).\hypertarget{index_overview}{}\section{Overview}\label{index_overview}
This project is a Voting System Software that allows many users who participate in the election of Open \hyperlink{classParty}{Party} List (\hyperlink{classCPL}{C\+PL}) and Closed \hyperlink{classParty}{Party} List (\hyperlink{classOPL}{O\+PL}) to produce election results. For Open \hyperlink{classParty}{Party} List ballet, voters express a preference for a particular candidate and party that each ballot will only have one candidate indicated as choice for the vote. For Closed \hyperlink{classParty}{Party} List ballet, voters express a preference for a particular party that the order of the \char`\"{}elected\char`\"{} officials is dependent on the order of the party\textquotesingle{}s numbering of teh candidates on the ballet.

In this system, the users can assume that there are no numbering error in the file, and the file will always be in the same directory as the program. In the Voting System Software, the users can enter the prompt command in the keyboard.\hypertarget{index_candidate}{}\section{Candidate}\label{index_candidate}
\hyperlink{classCandidate}{Candidate} represents a person who applies for election. In the candidate class, the system be able to access the information from the candidates in C\+LP and O\+Pl. Closed \hyperlink{classParty}{Party} List (\hyperlink{classCPL}{C\+PL}) contains the name of candidate and the priority party. Open \hyperlink{classParty}{Party} List (O\+Pl) contains only the name of candidate. \hyperlink{classCandidate}{Candidate}\textquotesingle{}s class provides more information about ballot numbers and the list of ballot id.\hypertarget{index_party}{}\section{Party}\label{index_party}
According to wikipedia.\+or, \hyperlink{classParty}{Party} is a political party that organized group of people who have the same ideology, or who otherwise have the same political positions, and who field candidates for elections, in an attempt to get them elected and thereby implement the party\textquotesingle{}s agenda (\href{https://en.wikipedia.org/wiki/Political_party}{\tt https\+://en.\+wikipedia.\+org/wiki/\+Political\+\_\+party}). In our Voting System, there are five parties, such as Democratic, Republican, Reform, Green, and Independent \hyperlink{classCandidate}{Candidate}. In the party\textquotesingle{}s class, the system can access the information for each parties with the list of candidates and the name of each party. In this class, the system be able to get access the number of of seats, number of votes and number of candidates. Also, we implement the functions that can add the candidate into the list of candidate and add the ballot id into the list of ballot id.\hypertarget{index_cpl}{}\section{C\+PL}\label{index_cpl}
\hyperlink{classCPL}{C\+PL} stands for Closed \hyperlink{classParty}{Party} List that the party fixes the order in which the candidates are listed and elected, and the voter simply casts a vote for the party as a whole. The voters are not be able to indicate their preference for any candidates on the list but must accept the list in the order presented by the party. Winning candidates are selected in the exact order they appear on the original list. In this class, we implement the Read\+File function which reads the file information into some corresponding object for \hyperlink{classCPL}{C\+PL} voting type. After reading the file, all the inormation are stored in report\+\_\+. In this class, we implement Populate\+Votes() function to count the votes. Also, we use struct sortbypartyvote to sort the order of the vote by comparing the number of votes. Whatever party gets the highest vote, it will appear at the top. Then, we implement the Distributre\+Seats() to get a total of seats. Also, we sort the priority order of the candidate. We implement Audit() function which counts votes and modify a party list into only containing the winners.\hypertarget{index_opl}{}\section{O\+PL}\label{index_opl}
O\+Pl stands for Open \hyperlink{classParty}{Party} List that voters express a preference for a particular candidate and a party. Each ballot will only have one candidate indicated as choice for the vote. In this class, we use similar method to \hyperlink{classCPL}{C\+PL} class. We implement the Read\+File function which reads the file information into some corresponding object for \hyperlink{classOPL}{O\+PL}. After reading the file, all the information will store in report\+\_\+. All other functions that we implement are similar to \hyperlink{classCPL}{C\+PL}, such as Read\+File(), struct sortbypartyvote, Audit(), Distribute\+Seats(), and Populate\+Votes(). The only different is, we implement struct sortbycandvote in \hyperlink{classOPL}{O\+PL}.\hypertarget{index_voting_type}{}\section{Voting\+Type}\label{index_voting_type}
Voting System allows many users who participate in the election for Open List \hyperlink{classParty}{Party} (\hyperlink{classOPL}{O\+PL}) and Closed \hyperlink{classParty}{Party} List (\hyperlink{classCPL}{C\+PL}) to produce election results with accurate, consistent, fair and helpful for all users. In this class, it contains general methods and variables which both \hyperlink{classOPL}{O\+PL} and \hyperlink{classCPL}{C\+PL} use. For example, both \hyperlink{classOPL}{O\+PL} and \hyperlink{classCPL}{C\+PL} use same \hyperlink{classReport}{Report} object, use same Split\+By\+Comma() method for reading information from file. In this class, we implement Read\+File() to read the information into some corresponding object depending on \hyperlink{classOPL}{O\+PL} or \hyperlink{classCPL}{C\+PL}. In this class, we randomly select the winner by using a unique number begin from 1. Then, we implement the Split\+By\+Comma() to split the string with delimiter comma \char`\"{},\char`\"{}.\hypertarget{index_voting_system}{}\section{Voting\+System}\label{index_voting_system}
\hyperlink{classVotingSystem}{Voting\+System} will call to run the method as main function runs. In this class, the user can input a file name for their voting system types, the user input will be save to this input\+\_\+file\+\_\+name\+\_\+. Also, we implement the Run() function to launches the system and starts to ask user to select options. Then, Run\+Audit\+Process() will run every steps of voting system. In this class, we implement some functions to handle the correct filename format and the directory to save the file.\hypertarget{index_report}{}\section{Report}\label{index_report}
\hyperlink{classReport}{Report} is used to report all the information related to election, such as candidates, party, voting type, number of ballots, number of seats, result of the election and winner. In this class, it will display all the election\textquotesingle{}s information for both \hyperlink{classOPL}{O\+PL} and \hyperlink{classCPL}{C\+PL} to the visible screen. Then, it exports both \hyperlink{classOPL}{O\+PL} and \hyperlink{classCPL}{C\+PL} to audit file by writing its into the file. After that, the report will export to the public file by sharing the report to other organization, such as Journalism, TV, and other official institutions as needed.

\begin{DoxyAuthor}{Author}

\begin{DoxyItemize}
\item Source code\+: U\+MN C\+SE Staffs
\item Editor\+: Team 22 
\end{DoxyItemize}
\end{DoxyAuthor}
